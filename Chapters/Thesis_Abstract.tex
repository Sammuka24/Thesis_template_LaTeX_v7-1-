%%%%%%%%%%%%%%%%%%%%%%%%%%%%%%%%%%%%%%%%%%%%%%%%%%%%%%%%%%%%%%%%%%%%%%%%
%                                                                      %
%     File: Thesis_Abstract.tex                                        %
%     Tex Master: Thesis.tex                                           %
%                                                                      %
%     Author: Andre C. Marta                                           %
%     Last modified :  2 Jul 2015                                      %
%                                                                      %
%%%%%%%%%%%%%%%%%%%%%%%%%%%%%%%%%%%%%%%%%%%%%%%%%%%%%%%%%%%%%%%%%%%%%%%%

\section*{Abstract}

% Add entry in the table of contents as section
\addcontentsline{toc}{section}{Abstract}
\vspace{50}
\hspace{10} Nowadays, with the evolution of manufacturing methods, namely additive manufacturing, several industries have studied the possibility of a possible change in their manufacturing processes. Additive manufacturing consists of depositing layer by layer and has numerous advantages, such as flexibility and geometric freedom or low costs at reduced production volumes when compared to traditional methods.\par
 Additive manufacturing in metals shows enormous potential in the aviation industry due to several advantages, including the reduction in weight of an aircraft, which is a constant challenge for engineers in this area. Among all the processes for manufacturing metal parts, forging stands out, as the most used conventional method over time. It is a well-known method that offers a reliable solution for the aerospace industry. However, with the complexity of the pieces increasing, the need for weight and punch reduction, emergence of additive manufacturing. Two most promising processes in additive manufacturing are direct energy deposition and powder bed fusion, which not only allows the construction of efficiently provided metal parts but also at complex component levels. \par
 The objective of this dissertation is to develop an economic and technological analysis of these manufacturing processes throughout their life cycle, creating a model where it is possible to select several production stages for a part, thus making this model unique for the study of additive manufacturing. It was also studied the impact that each technology has on the environment.\par
 Note that additive manufacturing is advantageous in allowing different geometries in the same production for lower volumes and that Direct Energy Deposition when compared to Powder Bed Fusion offers a faster and cheaper process, however the resolution quality of the pieces is smaller.

\vfill

\textbf{\Large Keywords:} Additive Manufacturing, Aerospace, Metal Parts, Forging, Life Cycle, Environmental impact

