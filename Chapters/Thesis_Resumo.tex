%%%%%%%%%%%%%%%%%%%%%%%%%%%%%%%%%%%%%%%%%%%%%%%%%%%%%%%%%%%%%%%%%%%%%%%%
%                                                                      %
%     File: Thesis_Resumo.tex                                          %
%     Tex Master: Thesis.tex                                           %
%                                                                      %
%     Author: Andre C. Marta                                           %
%     Last modified :  2 Jul 2015                                      %
%                                                                      %
%%%%%%%%%%%%%%%%%%%%%%%%%%%%%%%%%%%%%%%%%%%%%%%%%%%%%%%%%%%%%%%%%%%%%%%%

\section*{Resumo}

% Add entry in the table of contents as section
\addcontentsline{toc}{section}{Resumo}
\vspace{50}
\hspace{10} Hoje em dia, com a evolução dos métodos de fabrico, nomeadamente do fabrico aditivo, várias industrias estudam a possibilidade de uma eventual mudança dos seus processos de fabrico. O fabrico aditivo consiste em depositar camada por camada e possui enumeras vantagens, como flexibilidade e liberdade geométrica ou baixos custos a volumes de produção reduzidos quando comparados com os métodos tradicionais.\par
 O fabrico aditivo em metais mostra um enorme potencial na industria da aviação devido a varias vantagens, entre muitas a redução de peso de uma aeronave, que é um constante desafio dos engenheiros desta área. De entre todos os processos de fabrico de peças metálicas destaca-se o forjamento, como o método convencional mais utilizado ao longo dos tempo. É um método bastante conhecido que oferece uma solução fiável para a industria aeroespacial. No entanto, com a complexidade das peças a aumentar, a necessidade de redução de peso e de soques, surge o fabrico aditivo. Dois processos mais promissores da manufactura aditiva são o Direct Energy Deposition e o Powder Bed Fusion que não só permitem a construção de peças metálicas funcionais de forma eficiente como também a reparação de componentes complexas.\par
 O objectivo desta dissertação é desenvolver um analise económica e tecnológica destes processos de fabrico em todo o seu ciclo de vida, criando um modelo onde é possível seleccionar várias etapas de produção para uma peça, tornando assim este modelo único para o estudo do fabrico aditivo. Foi estudo ainda o impacto que cada tecnologia tem sobre o meio ambiente.\par
 Observa-se que a manufactura aditiva torna-se vantajosa ao permitir diferentes geometrias na mesma produção para volumes mais baixos e que o Direct Energy Deposition quando comparado com o Powder Bed Fusion oferece um processo mais rápido e ligeiramente mais barato, no entanto a qualidade de resolução das peças é menor.

\vfill

\textbf{\Large Palavras-chave:} Manufactura Aditiva, Aeroespacial, Peças Metálicas, Forjamento, Ciclo de Vida, Impacto Ambiental

