%%%%%%%%%%%%%%%%%%%%%%%%%%%%%%%%%%%%%%%%%%%%%%%%%%%%%%%%%%%%%%%%%%%%%%%%
%                                                                      %
%     File: Thesis_Introduction.tex                                    %
%     Tex Master: Thesis.tex                                           %
%                                                                      %
%     Author: Andre C. Marta                                           %
%     Last modified :  2 Jul 2015                                      %
%                                                                      %
%%%%%%%%%%%%%%%%%%%%%%%%%%%%%%%%%%%%%%%%%%%%%%%%%%%%%%%%%%%%%%%%%%%%%%%%

\chapter{Introduction}
\label{chapter:introduction}

% #############################################################################
\section{Overview}
\hspace{10} The Aerospace Manufacturer is a high technology industry that produces aircraft parts, designing, building, testing, selling and maintaining aircraft. Aviation has a major impact on a country’s economy, as far as connecting people, faster transport of goods and supplies and cost-effective are concerned \cite{camelia2010economic}. It has become essential for international trade and for tourism. Due to its importance in global trade, aerospace has a long history of technology inventions’, of new materials and of new sophisticated manufacturing processes which have been applied in other industries, decade after decade. In contrast to mass-production industries, aerospace industry has been focused towards complex and low-volume production \cite{synnes2016bridging}. It has been a constant challenge for production engineers due to constant challenges such as environmental performance restrictions, high manufacturing costs and competition market conditions.\par

Understanding modern aerospace manufacturing processes require that they are viewed in the context of historical development. In the beginning of aviation, only landing gear components and the main structures were entirely metallic. Skilled craftsmen were needed to machine the metals. This machining technique was called forging\cite{ASIM}. Being one of the oldest known metalworking processes that involves molding the material using compressive forces, this type of technique involves significant capital expenditures on machines, tools, installation and personnel, being a technology that offers low cost at a high production rate when Requested. \par

With the need to produce smaller, lighter, more complex parts and a low production volume, various techniques were researched and developed. With the advent of \ac{AM}, many of these problems have been solved. \ac{AM} is a parts manufacturing technique that builds 3D objects by adding layer by layer and can be used from various materials from plastic to metal. It is common for \ac{AM} technologies to use 3D molding software, where a \ac{CAD} sketch is then read by the 3D printer, which establishes and adds layer by layer successively, depositing the material in the form of liquid or powder.\par

Nowadays, \ac{AM} is used in a variety of industries, although it is still mostly used in the production of prototypes, a trend that is gradually being used in production processes. As mentioned earlier, production costs are a constant challenge for engineers. Developing a cost model can help aviation companies which is the best process to adopt for their production. \cite{rejeski2018research}\par

With aircraft engines becoming increasingly complex and sophisticated and the need to reduce aircraft weight, it became important to study and analyze new processes for the production of metal parts and consequently the method processing of metals, which are structural and engine components that represent a large part of the weight of the aircraft.\par 
This dissertation was developed in order to study the economic impact of various technologies in the production of a metal piece, from the construction to its post-processing, as well as the environmental impact that may be associated with. 



% #############################################################################
\section{Objectives}
The main goal of this work is to analyse the cost-effectiveness of forging process, and two additive manufacturing techniques most used in the production of metal parts,  \ac{PBF} and \ac{DED} process.\par
Thus, in order to find answers to this issue, the following objectives have been defined:
\begin{itemize}
    \item to understand the mechanisms of Forging, and AM technologies, \ac{PBF} and the \ac{DED}
    \item to analyze and determine all the costs along these three processes;
    \item to analyze the post-processing of each process.
    \item to perform a cost estimation model that estimates the costs of one piece;
    \item to apply the cost model to a case study and analyse the result obtained;
    \item to figure out how profitable is each process;
    \item to study the environmental impact of part production by each technology.
\end{itemize}




\section{Thesis Structure}
This thesis is organized as follows: 
\begin{itemize}
    \item Chapter 1 -  \emph{Introduction}- Provides a brief introduction on the topic of this work, the goals and also the structure of this dissertation;
    \item Chapter 2 - \emph{Theoretical Review} - Consists of a detailed review of all fundaments to frame this work. It is made a forging steps review, history, techniques, applications, industrial involvement and advantages and disadvantages. An introduction to the \ac{AM} process, including \ac{PBF} and \ac{DED} studied in this work and finishing processes used in \ac{AM} technology, as well.
    \item Chapter 3 - \emph{Methodology} - Explains how the theme for thid work came up, what research was done and how this cost model was developed.
    \item Chapter 4 - \emph{An Integrated Cost-Model} - Presents a developed cost model, together with an explanation of corresponding calculations, formulas needed and estimation timings to implement the cost estimation model.
    \item Chapter 5 - \emph{Results and Discussion} - Presents the results obtained from the cost model and an evaluation of that data with a complete discussion about the data obtained.
    \item Chapter  6 - \emph{Conclusion and Future Work} - Shows the current work and suggests future directions for further research in this topic.
\end{itemize}

